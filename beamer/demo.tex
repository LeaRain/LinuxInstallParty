\documentclass[9pt]{beamer}

%~~~~~~~~~~~~~~~~~~~~~~~~~~~~~~~~~~~~~~~~~~~~~~~~~~~~~~~~~~~~~~~~~~~~~~~~~~~~~~
% Use roboto Font (recommended)
\usepackage[sfdefault]{roboto}
\usepackage[utf8]{inputenc}
\usepackage[T1]{fontenc}
\usepackage[ngerman]{babel}
\usepackage{graphicx} 
\usepackage{hyperref}
%~~~~~~~~~~~~~~~~~~~~~~~~~~~~~~~~~~~~~~~~~~~~~~~~~~~~~~~~~~~~~~~~~~~~~~~~~~~~~~

%~~~~~~~~~~~~~~~~~~~~~~~~~~~~~~~~~~~~~~~~~~~~~~~~~~~~~~~~~~~~~~~~~~~~~~~~~~~~~~
% Define where theme files are located. ('/styles')
\usepackage{styles/fluxmacros}
\usefolder{styles}
% Use Flux theme v0.1 beta
% Available style: asphalt, blue, red, green, gray 
\usetheme[style=Asphaltf]{flux}
%~~~~~~~~~~~~~~~~~~~~~~~~~~~~~~~~~~~~~~~~~~~~~~~~~~~~~~~~~~~~~~~~~~~~~~~~~~~~~~

%~~~~~~~~~~~~~~~~~~~~~~~~~~~~~~~~~~~~~~~~~~~~~~~~~~~~~~~~~~~~~~~~~~~~~~~~~~~~~~
% Extra packages for the demo:
\usepackage{booktabs}
\usepackage{colortbl}
\usepackage{ragged2e}
\usepackage{schemabloc}
%~~~~~~~~~~~~~~~~~~~~~~~~~~~~~~~~~~~~~~~~~~~~~~~~~~~~~~~~~~~~~~~~~~~~~~~~~~~~~~

%~~~~~~~~~~~~~~~~~~~~~~~~~~~~~~~~~~~~~~~~~~~~~~~~~~~~~~~~~~~~~~~~~~~~~~~~~~~~~~
% Informations
\title{Linux Install Party}
\subtitle{\#AdoptYourOwnPenguin}
\author{Lea Laux}
\institute{Fachschaft Informatik Mathematik}
\date{4. Dezember 2019}
\titlegraphic{assets/tux.png}
%~~~~~~~~~~~~~~~~~~~~~~~~~~~~~~~~~~~~~~~~~~~~~~~~~~~~~~~~~~~~~~~~~~~~~~~~~~~~~~

\begin{document}

% Generate title page
\titlepage

\begin{frame}
 \frametitle{Inhalt}
 \tableofcontents
\end{frame}

\section{Einführung}
\subsection{Linux} 
\begin{frame}
 \frametitle{Linux?}
 \framesubtitle{Was ist das?}
	\begin{Large}
		\begin{itemize}
			\item Betriebssysteme basierend auf Linux-Kernel
	 		\item (Meist) quelloffene Systeme
	 		\item Vielseitig einsetzbar: Desktop, Server, Smartphones, Fernseher, Tablets, IoT-Geräte, ...
	 		\item Vorteile: Offene, kostenlose Systeme unter eigener Kontrolle, vielseitige Gestaltungsmöglichkeiten, ...
 		\end{itemize}
	\end{Large}
\end{frame}

\subsection{Distributionen}
\begin{frame}
	\frametitle{Ist das System relevant?}
	\framesubtitle{Verschiedene Distributionen}
	\begin{columns}
    \begin{column}[c]{0.6\textwidth}

	\begin{itemize}
		\item Vielzahl an Distributionen verfügbar
		\item Hier:
			\begin{itemize}
				\item Debian
				\item Manjaro
				\item Linux Lite
			\end{itemize}
		\item Beliebteste Distributionen: DistroWatch
		\item Welche Distribution passt zu mir? $\rightarrow$ DistroChooser
	\end{itemize}
	\end{column}
		\begin{column}[c]{0.4\textwidth}
	 \includegraphics[width=1.0\textwidth]{assets/distro.png}
	 Linux Lite Logo: CC BY-SA 4.0, modified, Jerry Bezencon (Creator of Linux Lite), \href{https://commons.wikimedia.org/wiki/File:Linux_Lite_Simple_Fast_Free_logo.png}{source}, License $\rightarrow$ end
    \end{column}
	\end{columns}
\end{frame}

\section{Systemrelevanz}
\subsection{Debian}
\begin{frame}
\frametitle{Debian}
\begin{columns}
    \begin{column}[c]{0.6\textwidth}

	\begin{itemize}
		\item Weit verbreitete Distribution, vor allem durch Derivate (über 480)
		\item Ins Leben gerufen: 1993
		\item Aktuell über 57000 verfügbare Programmpakete
		\item Etwa alle zwei Jahre neue Version (stable)
		\item Aktuelle Version: Debian 10 Buster
		\item oldoldstable -- oldstable -- \textbf{stable} -- testing -- unstable -- experimental
		\item Abgeschlossene Versionen mit möglichst wenig Updates
		\item Paketmanager: apt
	\end{itemize}
	\end{column}
	
	\begin{column}{0.4\textwidth}
    \includegraphics[width=1.0\textwidth]{assets/debian.png} 
	\end{column}

 \end{columns}

\end{frame}


\subsection{Manjaro}
\begin{frame}
 \frametitle{Manjaro}
 \begin{columns} 
    \begin{column}[c]{0.6\textwidth}

	\begin{itemize}
		\item Arch Linux-Derviat mit einem Fokus auf Benutzungsfreundlichkeit
		\item Ins Leben gerufen: 2011
		\item Rolling-Release-Modell
		\item \textbf{stable} -- testing -- unstable
		\item Durch Rolling Release oft Neuerungen und Zugriff auf aktuelle Software
		\item Zugriff auf Arch User Repository möglich
		\item Aktuelle Version: Juhraya 18.1
		\item Paketmanager: pacman (AUR helper/Pacman Wrapper yay)
	\end{itemize}
	\end{column}
	
	\begin{column}{0.4\textwidth}
    \includegraphics[width=1.0\textwidth]{assets/manjaro.png}
    \includegraphics[width=1.0\textwidth]{assets/arch.png} 
	\end{column}

 \end{columns}
\end{frame}

\subsection{Linux Lite}
\begin{frame}
 \frametitle{Linux Lite}
 \begin{columns} 
    \begin{column}[c]{0.6\textwidth}

	\begin{itemize}
		\item Ubuntu LTS (und damit auch Debian)-Derivat mit starkem Fokus auf Benutzungsfreundlichkeit
		\item Ins Leben gerufen: 2012
		\item Einstiegsfreundlichkeit: Welcome Center, Help Manual, Linux Lite Control Center 
		\item Sehr starke Anwendung einer graphischen Oberfläche
		\item Aktuelle Version: Diamond 4.6
		\item Paketmanager: apt
	\end{itemize}
	\end{column}
	
	\begin{column}{0.4\textwidth}
    \includegraphics[width=1.0\textwidth]{assets/linuxlite.png}
    Linux Lite Logo: CC BY-SA 4.0, Jerry Bezencon (Creator of Linux Lite), \href{https://commons.wikimedia.org/wiki/File:Linux_Lite_Simple_Fast_Free_logo.png}{source}, License $\rightarrow$ end
	\end{column}

 \end{columns}
\end{frame}

\section{Pinguin-Konfiguration}
\subsection{Graphische Oberfläche}
\begin{frame}
 \frametitle{Graphische Oberfläche}
 \framesubtitle{Es geht ums Aussehen!}
 \begin{Large}
 \begin{itemize}
  \item Verschiedenste Desktop-Umgebungen: KDE, XFCE, Gnome, Awesome, Cinnamon, MATE, ...
  \item Micro Management mit verschiedenen Themes, Optionen der verschiedenen Umgebungen, etc.
  \item Debian: Grundsätzlich keine Beschränkungen, gewisse Auswahl während (und nach) Installation, ausführlichere Liste: \href{https://wiki.debian.org/DesktopEnvironment}{https://wiki.debian.org/DesktopEnvironment}
  \item Manjaro: Offizielle und Community Editionen, Nachinstallation möglich, ausführlichere Liste: \href{https://manjaro.org/download/}{https://manjaro.org/download/}
  \item Linux Lite: Standardmäßig mit XFCE
 \end{itemize}
 \end{Large}
\end{frame}


\begin{frame}
 \centering
 \frametitle{Theme License}
 \framesubtitle{Flux}
 	\justifying
 Flux is a modern style beamer presentation. It is provided as a work in progress version and may suffer from inconsistencies. Sources and complementary information are available at\\[0.3cm]
 	\centering\textbf{github.com/pvanberg/flux-beamer}\\
 Flux is licensed under GNU General Public License v3.\\[0.3cm]
 	\centering\textbf{http://www.gnu.org/licenses}\\[0.3cm]
Inspired by \textbf{Metropolis} theme from Matthias Vogelgesang.\\
https://github.com/matze/mtheme 
 
\end{frame}

\end{document}
